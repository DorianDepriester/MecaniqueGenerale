\newcommand{\vect}[1]{\ensuremath{\overrightarrow{#1}}}

\newcommand{\R}[2]{%
	\glsdisp{Resu}{\vect{R}}\left(#1\rightarrow#2\right)
}
\newcommand{\Rc}{\gls{Rc}}

\newcommand{\M}[3]{%
	\glsdisp{Mome}{\vect{M}}_{#1}\left(#2\rightarrow#3\right)
}
\newcommand{\V}[3][\Rc]{%
	\glsdisp{V}{%
		\vect{V}}\left(\nicefrac{#2\in#3}{#1}\right)
}
\newcommand{\G}[3][\Rc]{%
	\glsdisp{G}{%
		\vect{\Gamma}}\left(\nicefrac{#2\in#3}{#1}\right)
}
\newcommand{\Om}[2][\Rc]{%
	\glsdisp{Omega}{%
		\vect{\Omega}\left(\nicefrac{#2}{#1}\right)
	}
}
\newcommand{\diff}[2][\Rc_0]{%
	\left.\frac{\mathrm{d}#2}{\mathrm{d}t}\right|_{#1}
}
\newcommand{\mcin}[3][\Rc]{%
	\glsdisp{sigma}{%
		\vect{\sigma}}_{#2}\left(\nicefrac{#3}{#1}\right)
}
\newcommand{\mdyn}[3][\Rc]{%
	\glsdisp{delta}{%
		\vect{\delta}}_{#2}\left(\nicefrac{#3}{#1}\right)
}

\newcommand{\Ec}[2][\Rc_0]{%
	\glsdisp{Ec}{%
		E_{\mathrm{c}}}\left(\nicefrac{#2}{#1}\right)
}
\newcommand{\torseur}[2]{%
	\left\lbrace\mathscr{#1}\left(#2\right)\right\rbrace
}
\newcommand{\tf}[2]{%
	\torseur{\glsdisp{tf}{F}}{#1\rightarrow #2}
}
\newcommand{\tv}[2][\Rc]{%
	\torseur{\glsdisp{tv}{V}}{\nicefrac{#2}{#1}}
}
\newcommand{\tc}[2][\Rc]{%
	\torseur{\glsdisp{tc}{C}}{\nicefrac{#2}{#1}}
}
\newcommand{\td}[2][\Rc]{%
	\torseur{\glsdisp{td}{D}}{\nicefrac{#2}{#1}}
}

\newcommand{\coords}[4][\Rc]{%
	\prescript{}{#1}{
		\left|
			\begin{array}{l}
				#2\\
				#3\\
				#4\\
			\end{array}
		\right.
	}
}
\newcommand{\inmat}[2][O]{%
	\glsdisp{inmat}{
		\uuline{I_{#1}\left(#2\right)}
	}
}
\newcommand{\incomp}[7][\Rc]{%
	\begin{bmatrix}
		#2		& -#7	& -#6\\
		-#7		& #3	& -#5\\
		-#6		& -#5	& #4	
	\end{bmatrix}_{#1}
}
\newcommand{\inope}[3][O]{%
	\glsdisp{inope}{%
		\vect{\mathcal{I}}_{#1}\left(#2,#3\right)
	}
}
\newcommand{\ints}[2][S]{%
	\int_{P\in #1}#2\mathrm{d}m
}

\newcommand{\puiss}[3][\Rc]{%
	\glsdisp{P}{\mathcal{P}}\left(#2\rightarrow\nicefrac{#3}{#1}\right)
}

\newcommand{\puissm}[2]{%
	\glsdisp{Pm}{\mathcal{P}}\left(#1\rightarrow #1\right)
}


\newcommand{\Ep}[3][\Rc]{%
	\glsdisp{P}{\mathrm{E}_{\mathrm{p}}}\left(#2\rightarrow\nicefrac{#3}{#1}\right)
}

\tikzset{cross/.style={cross out, draw=black, minimum size=2*(#1-\pgflinewidth), inner sep=0pt, outer sep=0pt},
%default radius will be 1pt. 
cross/.default={1pt}}


\newenvironment{note}{
	\itshape
	Note :
}{
}

\newcommand{\dm}{\mathrm{d}m}

\newtheorem{definition}{Définition}
\newcommand\numberthis{\addtocounter{equation}{1}\tag{\theequation}}

\newcommand{\hidden}[1]{%
	\begin{framed}
	\ifthenelse{%
		\equal{\ChoixDeVersion}{prof}
	 }{%
		#1
	 }{%
	 	{\color{white}
	 	#1
	 	}
	 }
	 \end{framed}
}

%\newenvironment{tors}[1]{%
%	\left\lbrace
%		\begin{array}{c}
%}{
%		\end{array}
%	\right\rbrace_{#1}	
%}
\NewDocumentEnvironment{tors}{m}{%
	\left\lbrace
		\begin{array}{c}
}{%
		\end{array}
	\right\rbrace_{#1}	
}