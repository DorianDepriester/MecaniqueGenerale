\chapter{Dynamique}


\section[Principe Fondamental de la Dynamique]{\gls{pfd}}
	\subsection{\'Enoncé}
	\label{sec:pfd}
\begin{theorem}
\hidden{
	Soit un ensemble matériel $E$. Il existe un repère \gls{Rg} tel que, pour tout sous-ensemble $e$ de $E$, le torseur dynamique de $e$ dans son mouvement par rapport à \gls{Rg} soit égal au torseur des actions extérieures à $e$ :
	\begin{equation}
		\forall e\subset E\quad \td[\gls{Rg}]{e}=\tf{\overline{e}}{e}
		\label{eq:pfd}
	\end{equation}
}
\end{theorem}
	Le repère \gls{Rg} est dit galiléen.
	
	\subsection{Théorème de la résultante}
	Du \gls{pfd}~\eqref{eq:pfd}, on en déduit le théorème de la résultante dynamique :
	\begin{equation}
		m\G[\gls{Rg}]{G}{e}=\R{\overline{e}}{e}
	\end{equation}

	\subsection{Théorème du moment}
	Du \gls{pfd}~\eqref{eq:pfd}, on en déduit le théorème du moment dynamique :
	\begin{equation}
		\forall A\quad \mdyn[\gls{Rg}]{A}{e}=\M{A}{\overline{e}}{e}
	\end{equation}

\section{Théorème des actions mutuelles}
\label{sec:actions-mutuelles}
\begin{theorem}
	\hidden{
	Soient deux ensembles de solides $(e_1)$ et $(e_2)$ distincts. L'action mécanique de $(e_1)$ sur $(e_2)$ est opposée à l'action mécanique de $(e_2)$ sur $(e_1)$ :
	\begin{equation}
		\tf{e_1}{e_2}=-\tf{e_2}{e_1}
	\end{equation}
	}
\end{theorem}
\begin{proof}
Le \gls{pfd} appliqué à l'ensemble $(e_1)$ nous donne :
\begin{equation}
	\td[\gls{Rg}]{e_1}=\tf{\overline{e_1}}{{e_1}}
	\label{eq:pdf-e1}
\end{equation}
On pose :
\begin{equation}
	E=e_1\cup e_2
	\label{eq:e1Ue2}
\end{equation}
On a donc :
\begin{equation*}
	\overline{e_1}=e_2\cup\overline{E}
\end{equation*}
Ainsi, \eqref{eq:pdf-e1} s'écrit alors :
\begin{equation}
	\td[\gls{Rg}]{e_1}=\tf{{e_2}}{{e_1}}+\tf{\overline{E}}{{e_1}}
	\label{eq:pdf-e1-end}
\end{equation}
Par le même raisonnement, on a :
\begin{equation}
	\td[\gls{Rg}]{e_2}=\tf{{e_1}}{{e_2}}+\tf{\overline{E}}{{e_2}}
	\label{eq:pdf-e2-end}	
\end{equation}
On trouve donc, d'après \eqref{eq:pdf-e1-end} et \eqref{eq:pdf-e2-end} :
\begin{equation}
	\td[\gls{Rg}]{e_1}+\td[\gls{Rg}]{e_2}=\tf{{e_2}}{{e_1}}+\tf{\overline{E}}{{e_1}}+\tf{{e_1}}{{e_2}}+\tf{\overline{E}}{{e_2}}
	\label{eq:pfd-tot}
\end{equation}
D'après la définition \eqref{eq:e1Ue2}, on a :
\begin{subequations}
	\begin{align}
		\td[\gls{Rg}]{e_1}+\td[\gls{Rg}]{e_2}&=\td[\gls{Rg}]{E}\\
		\tf{\overline{E}}{{e_1}}+\tf{\overline{E}}{{e_2}}&=\tf{\overline{E}}{{E}}
	\end{align}
\end{subequations}
Or, en appliquant le \gls{pfd} à $E$, on trouve :
\begin{equation*}
	\td[\gls{Rg}]{E}=\tf{\overline{E}}{{E}}
\end{equation*}
En simplifiant~\eqref{eq:pfd-tot}, on obtient donc :
\begin{equation*}
	\left\lbrace 0\right\rbrace=\tf{{e_2}}{{e_1}}+\tf{{e_1}}{{e_2}}
\end{equation*}
\end{proof}

\section{Cas des repères non galiléens}
\begin{theorem}
	Soient $e$ un ensemble matériel, \gls{Rc} un repère quelconque et \gls{Rg} un repère galiléen. Le \gls{pfd} appliqué à $e$ nous donne alors :
	\begin{equation}
		\td{e}=\tf{\overline{e}}{e}
			+\left\lbrace\mathcal{D}_{\mathrm{ie}}\left(e,\nicefrac{\gls{Rc}}{\gls{Rg}}\right)\right\rbrace
			+\left\lbrace\mathcal{D}_{\mathrm{ic}}\left(e,\nicefrac{\gls{Rc}}{\gls{Rg}}\right)\right\rbrace
		\label{eq:pfd-Rc}
	\end{equation}
	Avec $\left\lbrace\mathcal{D}_{\mathrm{ie}}\left(e,\nicefrac{\gls{Rc}}{\gls{Rg}}\right)\right\rbrace$ le torseur des effets d'inertie d'entrainement et $\left\lbrace\mathcal{D}_{\mathrm{ic}}\left(e,\nicefrac{\gls{Rc}}{\gls{Rg}}\right)\right\rbrace$ le torseur des effets d'inertie de Coriolis :
\begin{subequations}
	\begin{align}
		\left\lbrace\mathcal{D}_{\mathrm{ie}}\left(e,\nicefrac{\gls{Rc}}{\gls{Rg}}\right)\right\rbrace&=
		\left\lbrace
			\begin{array}{>{\displaystyle}c}
				-\ints[e]{\G{P}{e}}\\
				-\ints[e]{\vect{AP}\wedge\G{P}{e}}
			\end{array}
		\right\rbrace_A\\
		\left\lbrace\mathcal{D}_{\mathrm{ic}}\left(e,\nicefrac{\gls{Rc}}{\gls{Rg}}\right)\right\rbrace&=
		\left\lbrace
			\begin{array}{>{\displaystyle}c}
				-\ints[e]{2\Om[\gls{Rg}]{\Rc}\wedge\V{P}{e}}\\
				-\ints[e]{\vect{AP}\wedge\left[2\Om[\gls{Rg}]{\Rc}\wedge\V{P}{e}\right]}
			\end{array}
		\right\rbrace_A			
	\end{align}	 
\end{subequations}
\end{theorem}
\begin{proof}
D'après la composition des accélérations (eq.~\ref{eq:compo-acceleration-th} p.~\pageref{eq:compo-acceleration-th}), on a :
\begin{equation*}
	\G[\gls{Rg}]{P}{e}=\G[\gls{Rc}]{P}{S}+\G[\gls{Rg}]{P}{\Rc}+2\Om[\gls{Rg}]{\Rc}\wedge\V[\gls{Rc}]{P}{S}
\end{equation*}
On a donc :
\begin{equation*}
	\td[\gls{Rg}]{e}=
		\left\lbrace
			\begin{array}{>{\displaystyle}c}
				\ints[e]{\left[\G[\gls{Rc}]{P}{e}+\G[\gls{Rg}]{P}{\Rc}+2\Om[\gls{Rg}]{\Rc}\wedge\V[\gls{Rc}]{P}{e}\right]}\\
				\ints[e]{\vect{AP}\wedge\left[\G[\gls{Rc}]{P}{e}+\G[\gls{Rg}]{P}{\Rc}+2\Om[\gls{Rg}]{\Rc}\wedge\V[\gls{Rc}]{P}{e}\right]}
			\end{array}
		\right\rbrace_A
\end{equation*}
\begin{equation*}
	 	\begin{split}
	 		=
			\left\lbrace
				\begin{array}{>{\displaystyle}c}
					\ints[e]{\G[\gls{Rc}]{P}{e}}\\
					\ints[e]{\vect{AP}\wedge\G[\gls{Rc}]{P}{e}}
				\end{array}
			\right\rbrace_A
			+
			\left\lbrace
				\begin{array}{>{\displaystyle}c}
					\ints[e]{\G[\gls{Rg}]{P}{\gls{Rc}}}\\
					\ints[e]{\vect{AP}\wedge\G[\gls{Rg}]{P}{\gls{Rc}}}
				\end{array}
			\right\rbrace_A\\
			+
			\left\lbrace
				\begin{array}{>{\displaystyle}c}
					\ints[e]{2\Om[\gls{Rg}]{\Rc}\wedge\V[\gls{Rc}]{P}{e}}\\
					\ints[e]{\vect{AP}\wedge\left[2\Om[\gls{Rg}]{\Rc}\wedge\V[\gls{Rc}]{P}{e}\right]}
				\end{array}
			\right\rbrace_A		
		\end{split}
\end{equation*}
Or :
\begin{equation*}
	\left\lbrace
		\begin{array}{>{\displaystyle}c}
			\ints[e]{\G[\gls{Rc}]{P}{e}}\\
			\ints[e]{\vect{AP}\wedge\G[\gls{Rc}]{P}{e}}
		\end{array}
	\right\rbrace_A
	=
	\td{e}
\end{equation*}
et d'après le \gls{pfd} \eqref{eq:pfd} appliqué à $e$ :
\begin{equation*}
	\td[\gls{Rg}]{e}=\tf{\overline{e}}{e}
\end{equation*}
On a donc :
\begin{equation*}
	\tf{\overline{e}}{e}=\td{e}
	-\left\lbrace\mathcal{D}_{\mathrm{ie}}\left(e,\nicefrac{\gls{Rc}}{\gls{Rg}}\right)\right\rbrace
	-\left\lbrace\mathcal{D}_{\mathrm{ic}}\left(e,\nicefrac{\gls{Rc}}{\gls{Rg}}\right)\right\rbrace
\end{equation*}
\end{proof}